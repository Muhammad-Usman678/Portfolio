\documentclass[pt,a4paper]{article}
\usepackage[utf8]{inputenc}
\usepackage[margin=0.75in]{geometry}
\usepackage{array,tabularx,enumitem}
\usepackage[hidelinks,breaklinks]{hyperref}
\usepackage{graphicx}

\usepackage{url}
\def\UrlBreaks{\do\/\do-}
\newcommand{\pubitem}[2]{%
  \noindent\parbox[t]{0.9\textwidth}{#1}\hfill #2\par\vspace{0.4em}
}
\usepackage[document]{ragged2e}
\usepackage{fontawesome5}
\usepackage{graphicx}
\newcommand{\sectiontitle}[1]{%
  \vspace{4pt}\noindent\textbf{\large #1}%
  \leavevmode\leaders\hrule height 0.3pt \hfill\kern0pt%
  \vspace{1pt}\par
}

\newcommand{\jobtitle}[1]{\noindent\textbf{#1}}
\newcommand{\companylocation}[2]{\noindent\textbf{#1} \hfill #2\newline}
\setlist[itemize]{noitemsep,topsep=0pt,leftmargin=15pt}

\begin{document}
\noindent
\raisebox{45pt}{%
\begin{minipage}[t]{0.72\textwidth}
{\LARGE \bfseries Muhammad Usman}\\[3pt]
Bruchwiesenanlage 4, 66125, Saarbruecken, Germany \textbar{} +49 174 276 7793 \textbar{}\\
\href{mailto:usmanchaudhary.eng@gmail.com}{usmanchaudhary.eng@gmail.com} \textbar{}
\href{https://www.linkedin.com/in/usmanbioinformatician2025/}{LinkedIn} \textbar{}
\href{https://github.com/Muhammad-Usman678}{GitHub}
\end{minipage}
}
\hfill
\begin{minipage}[t]{0.23\textwidth}
\raggedleft
\includegraphics[width=2.6cm]{image_cv.png}
\end{minipage}








\sectiontitle{Professional Summary}
Bioinformatician and \textbf{DevOps-oriented Machine Learning Engineer} with expertise building scalable solutions in \textbf{genomics, transcriptomics, and proteomics}. In the final phase of a Master’s degree exploring opportunities at the intersection of biological data analysis, machine learning, and infrastructure. Focused on building reproducible, scalable pipelines that translate complex biological data into insight.



\sectiontitle{Education}

\textbf{Bioinformatics, M.Sc.} \hfill Oct 2022 -- Present \\
Saarland University, Germany
\begin{itemize}
\item Thesis: \emph{ML-Driven Exploration of NRPS Engineering in \textit{E. coli}}
\item Focus: Computational Biology, Genomics, NGS, Machine Learning
\end{itemize}

\textbf{Biomedical Engineering, B.Sc.} \hfill Oct 2018 -- Aug 2022 \\
University of Engineering and Technology (UET), Lahore, Pakistan
\begin{itemize}
\item Graduated with Honors (GPA: 3.72/4.00)
\item Thesis: \emph{ECG Analysis: Monitoring of Signs \& Symptoms}
\end{itemize}


\sectiontitle{Work Experience}


\textbf{AI Safety Research Fellow} \hfill Nov 2025 -- Present \\
\textbf{Saarland University, Germany}
\begin{itemize}[nosep, leftmargin=*]
    \item Developing \textbf{multi-omics ML models} for personalized oncology using fusion and graph-based approaches.
    \item Assessing \textbf{uncertainty, causality, and robustness} under cohort and distribution shifts.
    \item Building reproducible \textbf{Snakemake + Docker} pipelines with AI-safety methods for reliable benchmarking.
\end{itemize}


\vspace{0.5em}
\textbf{Machine Learning \& Bioinformatics Engineer} \hfill Jun 2024 -- Present \\
\textbf{Myria Biosciences, Switzerland}

\begin{itemize}
\item Processed and analyzed \textbf{10,000+ genomic and proteomic samples}, reducing pipeline runtime by \textbf{35\%} using workflow managers and HPC clusters.
\item Benchmarked variant calling pipelines, improving SNP detection sensitivity by \textbf{8\%} against reference datasets.
\item Designed and automated \textbf{ETL, NGS, and ML pipelines} (\textbf{Python, Pandas, SQL, Docker, CI/CD}), ingesting and validating \textbf{50 GB/week} of multi-omics data with \textbf{\textless{}2\% error rate}.
\item Built a \textbf{graph-based knowledge system (Neo4j)} integrating \textbf{10,000+ biological entities}, improving cross-domain query performance by \textbf{50\%}.
\item Delivered \textbf{30+ interactive dashboards and reproducible analytical reports}, version-controlled and containerized with \textbf{Git and Docker}, supporting decisions across \textbf{3 clinical programs}.
\end{itemize}





\vspace{0.5em}

\textbf{Research Assistant} \hfill Oct 2024 -- Jul 2025 \\
\textbf{Saarland University, Germany}

\begin{itemize}
\item Applied \textbf{AI-driven CADD methodologies} for biosynthetic pathway optimization and molecule screening.
\item Integrated synthetic biology with computational chemistry workflows for rapid in-silico experimentation.
\item Automated ML pipelines using \textbf{Python, PyTorch, and Scikit-learn} on Linux, improving model testing and reproducibility.
\end{itemize}

\vspace{0.5em}
\textbf{Data Scientist in Drug Bioinformatics} \hfill Jun 2024 -- Dec 2024 \\
\textbf{Helmholtz Institute (HIPS), Germany}
\begin{itemize}
\item Integrated NGS and \textbf{LC-MS/MS proteomics} data, enabling discovery of \textbf{15 novel BGCs}.
\item Developed \textbf{machine learning models} improving expression-efficiency prediction accuracy by \textbf{22\%}.
\item Applied \textbf{graph-based methods} for pathway discovery, with \textbf{experimentally validated predictions}.
\item Built version-controlled, reproducible \textbf{Snakemake pipelines} adopted by multiple research groups, saving \textbf{40+ hours/month}.
\end{itemize}


\vspace{0.5em}

\textbf{Research Assistant} \hfill Dec 2022 -- Sep 2024 \\
\textbf{Korean Institute of Science \& Technology (KIST) Europe, Germany}

\begin{itemize}
\item Supported sequencing-driven synthetic biology projects, validating \textbf{100+ cloning experiments}.
\item Led student training and workflow standardization, reducing onboarding time by \textbf{30\%}.
\item Executed and analyzed \textbf{ELISA, cloning workflows, and phage–host interaction} studies to support bacteriophage engineering.
\end{itemize}

\sectiontitle{Volunteering}

\textbf{Research Mentor, Computational Biology} \hfill Sep 2022 -- Present \\
\textbf{UET Lahore, Pakistan}
\begin{itemize}
\item Mentored undergraduate research projects applying \textbf{machine learning and deep learning} to biological and biomedical data (EEG, genomic, physiological).
\item Guided students in research design, data preprocessing, feature engineering, and model evaluation, strengthening scientific rigor and outcomes.
\item Provided hands-on technical guidance in \textbf{Python, scikit-learn, PyTorch}, emphasizing reproducible and research-grade coding practices.
\end{itemize}

\sectiontitle{Projects \& Certifications}
\begin{itemize}[leftmargin=*, nosep, itemsep=4pt]
\item \textbf{GeneCoViz}, Co-expression network visualization (Plotly)
\item \textbf{NeuroVision Explorer}, MRI analysis using CNNs/transformers
\item \textbf{Heart Disease Prediction}, \textbf{92\% accuracy} (RF/XGBoost)
\item \textbf{Neo4j Knowledge Graph}, 10k+ biological entities
\item \textbf{FastAPI CI/CD Reference}, Automated Docker deployment
\item Neo4j Certified Professional; Graph Data Science Certified
\item AWS Cloud Practitioner Essentials
\end{itemize}

\sectiontitle{Publications}
\pubitem{Reprogramming Filamentous FD Viruses to Capture Copper Ions. \textit{ChemBioChem}. DOI: \href{https://doi.org/10.1002/cbic.202400237}{10.1002/cbic.202400237}}{2024}
\pubitem{Bacteriophage Engineering for Improved Copper Ion Binding. \textit{Macromolecular Bioscience}. DOI: \href{https://doi.org/10.1002/mabi.202300354}{10.1002/mabi.202300354}}{2023}
\pubitem{Electrocardiogram Data Visualization and Dimensionality Reduction. \href{https://fke.utm.my/icbhs2022/wp-content/uploads/sites/10/2022/10/JOINT-CONFERENCE-BOOK_compressed.pdf}{\textit{ICBHS 2022}}}{2022}
\pubitem{Brain Tumor Segmentation and Detection on MRI Images. \href{https://fke.utm.my/icbhs2022/wp-content/uploads/sites/10/2022/10/JOINT-CONFERENCE-BOOK_compressed.pdf}{\textit{ICBHS 2022}}}{2022}

\sectiontitle{Awards \& Hackathons}

\noindent\textbf{Winner, Health Hack Saar} \hfill Oct 2025 \\
\textit{Saarbrücken, Germany} \\
\textbf{MEDPlex: Bridging the Gender Data Gap in Pediatrics}
\begin{itemize}
    \item Built a real-time \textbf{evidence-based pediatric medication recommendation system} addressing gender bias in dosing.
    \item Delivered a full working prototype within 48 hours with potential to improve medication safety and reduce dosing errors.
\end{itemize}

\noindent\textbf{Finalist, ODDO BHF Equity Research Hackathon} \hfill Nov 2025 \\
\textit{Paris, France} \\
\textbf{Enhancing Equity Research Client Engagement}
\begin{itemize}[nosep, leftmargin=*]
    \item Selected among top teams and invited to pitch at \textbf{ODDO BHF Headquarters} in Paris.
    \item Built an AI-augmented platform improving analyst–client engagement through availability signalling and automated follow-ups.
\end{itemize}

\noindent\textbf{Participant, Defensive Acceleration Hackathon} \hfill Nov 2025 \\
\textit{Global} \\
\textbf{BioCast AI: Global Disease Outbreak Forecasting}
\begin{itemize}[nosep, leftmargin=*]
    \item Developed a hybrid \textbf{LSTM + XGBoost} epidemiological forecasting pipeline with PCA/clustering analytics and outbreak risk prediction.
\end{itemize}





\sectiontitle{Skills}

\textbf{NGS \& Bioinformatics:} WGS, WES, RNA-seq (bulk \& single-cell), methylome, variant calling, annotation, Bioconductor \\
\textbf{ML \& Data Science:} Scikit-learn, PyTorch, PCA/UMAP/t-SNE, clustering, Random Forest, SVM, XGBoost, SHAP, GNNExplainer, graph ML \\
\textbf{DevOps \& Infrastructure:} Docker, Git, Bash, Linux Server Management, CI/CD, Workflow Automation, Cloud Deployment \\
\textbf{Pipelines \& Tools:} Snakemake, Nextflow, BioPython, Pandas, NumPy, Jupyter, Flask \\
\textbf{Data Engineering \& Databases:} SQL, ETL/ELT, API pipelines, Neo4j knowledge graphs, data modeling \\
\textbf{Visualization:} Plotly, R Shiny, Jupyter Dashboards, RMarkdown, interactive reporting \\
\textbf{Languages:} English, German \\




\end{document}
